\documentclass{beamer}
\usetheme{Madrid}

\usepackage{listings}


\title{Local Search Strategies for Satisfiability Testing}
\subtitle{GSAT using Random Walk Strategy}
\author{by Oriol Alàs and Joaquim Picó}
\centering
\date{April 2020}
\begin{document}
\maketitle
\begin{frame}
\frametitle{Table of Contents}
\tableofcontents
\end{frame}

\begin{frame}{Introduction}
\framesubtitle{Why have we chosen this paper?}
\section{Introduction}
\begin{itemize}
	\item Interest in:
	\begin{itemize}
		\item  learning some new strategies.
		\item finding out better implementations of GSAT or WalkSAT for our SAT solver.
	\end{itemize}
	\item Wanted to know strategies that mixes the given ones in class.
	\item It talks about simulated annealing and GWSAT strategy.
\end{itemize}
\end{frame}
\begin{frame}{GSAT}
\framesubtitle{Characteristics}
\section{GSAT}
\subsection{Characteristics}
\begin{itemize}
	\item Strategy used for incomplete solvers.
    \item Searches for the "most" satisfiable interpretation. The interpretation that has the maximum number of clauses evaluated to true.
\end{itemize}
\end{frame}
\begin{frame}[fragile]
\subsection{Algorithm}
\begin{verbatim}
for i:=1 to MAX-TRIES
   I := random interpretation for F
   for j:=1 to MAX_FLIPS
      if I satisfies then return I
      Flip any variable in I that
         results in greatest decrease
         in the number o unsatisfied clauses
   end for
end for
return No SAT
\end{verbatim}
\end{frame}
\begin{frame}{GSAT}
\framesubtitle{Advantages and drawbacks}
\subsection{Advantages and drawbacks}
\begin{itemize}
	\item It changes only those literals which have a better number of unsatisfied clause. 
	\item The solver can get stuck in a local optimal in every flip. There are some variations of GSAT to avoid this:
	\begin{itemize}
		\item Using \textbf{restarts}. Creating a new random interpretation every $n$ loop turns.
		\item \textbf{Tabu Search}. Using a data structure that contains all the local optimal already visited. It can be powerful although a high amount of memory is needed in large CNF formulas.
		\item \textbf{GWSAT}.
	\end{itemize}
\end{itemize}
\end{frame}

\begin{frame}{Simulated Annealing}
\section{Simulated Annealing}
\end{frame}

\begin{frame}{Random Walk GSAT}
\framesubtitle{Algorithm}
\section{Random Walk GSAT}
\subsection{Algorithm}
\end{frame}

\begin{frame}{Random Walk GSAT}
\framesubtitle{Results}
\subsection{Results}
\end{frame}


\begin{frame}{Modifying GWSAT}
\section{Modifying GWSAT}
\end{frame}

\begin{frame}{Conclusions}
\section{Conclusions}
\end{frame}
\end{document}
